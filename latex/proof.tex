\documentclass [12pt, a4paper] {article}
\usepackage[utf8x]{inputenc}
\usepackage[english,russian]{babel}
\usepackage[T2A]{fontenc}
\usepackage{hyperref}
\usepackage{mathtools, amsthm}
\usepackage {graphicx}
%\let\stdsection\section
%\renewcommand\section{\newpage\stdsection}
\addto\captionsrussian{% Replace "english" with the language you use
  \renewcommand{\contentsname}%
    {Оглавление}%
}
\hypersetup{
    colorlinks,
    allcolors=blue
}

\newtheorem{myth}{Теорема}
\newtheorem{mylm}{Лемма}

\begin {document}

\section{Разложение функций}
Рассмотрим функции
\[ \begin{array}{l}
    h = hx^{k-1} + tx^{k-2} + \dots + tx + t \\
    t = tx^{k-1} + ehx^{k-2} + \dots + ehx + eh \text{, где}
\end{array} \]
$e$ -- некоторый коэффициент.

Через $C^{hd}_{hi}$ обозначим коэффициент у функции $h$ при $x^i$, при 
функции $h$, поляризация $d$.

Через $C^{hd}_{ti}$ обозначим коэффициент у функции $h$ при $x^i$, при 
функции $t$, поляризация $d$.

Через $C^{td}_{hi}$ обозначим коэффициент у функции $t$ при $x^i$, при 
функции $h$, поляризация $d$.

Через $C^{td}_{ti}$ обозначим коэффициент у функции $t$ при $x^i$, при 
функции $t$, поляризация $d$.

\[ C^{hd}_{hi} = \binom{k-1}{i}(-d)^{k-1-i} \]
\[ C^{hd}_{ti} = \sum_{j=i}^{k-2} \binom{j}{i}(-d)^{j-i} \]
\[ C^{td}_{hi} = \sum_{j=i}^{k-2} e \binom{j}{i}(-d)^{j-i} \]
\[ C^{td}_{ti} = \binom{k-1}{i}(-d)^{k-1-i} \]

Заметим, что $\binom{k-1}{i} \neq 0 \pmod k$, тогда у функции $h$ при любой
поляризации присутствует слагаемое с $h$, а у функции $t$ при любой поляризации
присутствует слагаемое с $t$.

Пусть $f_a = h + at$, где $a \in [1 .. k-1]$.

\begin{myth}
    Для любых $d$ и $a$ у полинома функции $f^{(d)}_a$ $k$ слагаемых, если $e$
    -- квадратичный невычет по модулю $k$.
\end{myth}
\begin{proof}
    Пусть существуют $a, d, i$ такие, что $f^{(d)}_a[i] = 0$, тогда
    $C^{hd}_{hi}$ должно быть равно $-a C^{td}_{hi}$, а $C^{hd}_{ti}$ должно
    быть равно $-a C^{td}_{ti}$.
    \begin{equation*}
        \begin{cases}
            \binom{k-1}{i}(-d)^{k-1-i} & = \; 
                -a \sum\limits_{j=i}^{k-2} e \binom{j}{i} (-d)^{j-i} \\
            \sum\limits_{j=i}^{k-2} \binom{j}{i} (-d)^{j-i} & = \;
                -a \binom{k-1}{i}(-d)^{k-1-i} 
        \end{cases}
    \end{equation*} 
    $\sum\limits_{j=i}^{k-2} \binom{j}{i} (-d)^{j-i} \neq 0$ так как
    $\binom{k-1}{i}(-d)^{k-1-i} \neq 0$.
    Следовательно 
    \[
        a^{-1} \sum_{j=i}^{k-2} \binom{j}{i} (-d)^{j-i} =
            a \sum_{j=i}^{k-2} e \binom{j}{i} (-d)^{j-i}
    \]

    Значит $e = (a^{-1})^2$, что противоречит с тем, что $e$ -- квадратичный
    невычет по модулю $k$.
\end{proof}

\section{Функции одной переменной}
Рассмотрим теперь следующие функции $h$ и $t$ одной переменной:
\[
    \begin{array}{l}
        h = ex^{k-1} + (x-1)^{k-1} \\
        t = x^{k-1} + e(x-1)^{k-1}
    \end{array}
\]

Они выглядят так:
\[
    \begin{array}{l}
        h = (e+1)x^{k-1} + x^{k-2} + \dots + x + 1\\
        t = (e+1)x^{k-1} + ex^{k-2} + \dots + ex + e
    \end{array}
\]

Через $C_i^{hd}$ обозначим коэффициент при $x^i$ у функции $h$ при поляризации
$d$.

Через $C_i^{td}$ обозначим коэффициент при $x^i$ у функции $t$ при поляризации
$d$.

\[
    \begin{array}{l}
        C_i^{hd} = \binom{k-1}{i} (-d)^{k-1-i} +
            \sum\limits_{j=i}^{k-2} e \binom{j}{i} (-d)^{j-i} \\
        C_i^{hd} = e \binom{k-1}{i} (-d)^{k-1-i} +
            \sum\limits_{j=i}^{k-2} \binom{j}{i} (-d)^{j-i}
    \end{array}
\]

\begin{myth}
    Для любой поляризации $d$, для любого $i$ у любой пары $f$ и $g$ функций из
    $\{h, t, h + at\}$ $(\forall a \in [1..k-1])$ коэффициенты при $x^i$ не
    могут быть равны 0 одновременно.
\end{myth}
\begin{proof}
    
    1) Рассмотрим случай, когда $f$ и $g$ $\in \{h, t\}$.
    Предположим, что $C_i^{hd} = 0$ и $C_i^{td} = 0$. Тогда, так как
    $\binom{k-1}{i} (-d)^{k-1-i} \neq 0$, то и $\sum\limits_{j=i}^{k-2}
    \binom{j}{i} (-d)^{j-i} \neq 0$. А также 
    \[
        e \sum\limits_{j=i}^{k-2} \binom{j}{i} (-d)^{j-i} = 
        e^{-1} \sum\limits_{j=i}^{k-2} \binom{j}{i} (-d)^{j-i}
    \]
    Следовательно $e^2 = 1$, но $e \in [2..k-2]$ чего не может быть так как
    у 1 только два корня 1 и -1.

    2) Пусть $f = h + at$, а $g \in \{h, t\}$.
    \[
        C_i^{fd} = C_i^{hd} + aC_i^{td} \text{.}
    \]
    Если $C_i^{gd} = 0$, то в $C_i^{fd}$ всего 1 слагаемое и, в силу
    предыдущего пункта, оно отлично от 0.

    3) Последний случай: $f = h + at$, $g = h + bt$.
    \[
        \begin{array}{l}
            C_i^{fd} = C_i^{hd} + aC_i^{td} \text{,} \\
            C_i^{gd} = C_i^{hd} + bC_i^{td} \text{.}
        \end{array}
    \]
    Если $C_i^{td} \neq 0$, то $C_i^{fd} \neq C_i^{gd}$, а если $C_i^{td} = 0$,
    то из первого пункиа следует, что $C_i^{hd} \neq 0$.

\end{proof}

\end {document}
