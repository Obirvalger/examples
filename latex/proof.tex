\documentclass [12pt, a4paper] {article}
\usepackage[utf8x]{inputenc}
\usepackage[english,russian]{babel}
\usepackage[T2A]{fontenc}
\usepackage{hyperref}
\usepackage{mathtools, amsthm}
\usepackage {graphicx}
%\let\stdsection\section
%\renewcommand\section{\newpage\stdsection}
\addto\captionsrussian{% Replace "english" with the language you use
  \renewcommand{\contentsname}%
    {Оглавление}%
}
\hypersetup{
    colorlinks,
    allcolors=blue
}
\newtheorem{myth}{Теорема}

\begin {document}

\section{Теорема}
Рассмотрим функции
\[ \begin{array}{l}
    h = hx^{k-1} + tx^{k-2} + \dots + tx + t \\
    t = tx^{k-1} + ehx^{k-2} + \dots + ehx + eh \text{, где}
\end{array} \]
$e$ -- некоторый коэффициент.

Через $C^{hd}_{hi}$ обозначим коэффициент у функции $h$ при $i$ слагаемом, при 
функции $h$, поляризация $d$.

Через $C^{hd}_{ti}$ обозначим коэффициент у функции $h$ при $i$ слагаемом, при 
функции $t$, поляризация $d$.

Через $C^{td}_{hi}$ обозначим коэффициент у функции $t$ при $i$ слагаемом, при 
функции $h$, поляризация $d$.

Через $C^{td}_{ti}$ обозначим коэффициент у функции $t$ при $i$ слагаемом, при 
функции $t$, поляризация $d$.

\[ C^{hd}_{hi} = \binom{k-1}{i}(-d)^{k-1-i} \]
\[ C^{hd}_{ti} = \sum_{j=i}^{k-2} \binom{j}{i}(-d)^{j-i} \]
\[ C^{td}_{hi} = \sum_{j=i}^{k-2} e \binom{j}{i}(-d)^{j-i} \]
\[ C^{td}_{ti} = \binom{k-1}{i}(-d)^{k-1-i} \]

Заметим, что $\binom{k-1}{i} \neq 0 \pmod k$, тогда у функции $h$ при любой
поляризации присутствует слагаемое с $h$, а у функции $t$ при любой поляризации
присутствует слагаемое с $t$.

Пусть $f_a = h + at$, где $a \in [1 .. k-1]$.

\begin{myth}
    Для любых $d$ и $a$ у полинома функции $f^{(d)}_a$ $k$ слагаемых, если $e$
    -- квадратичный невычет по модулю $k$.
\end{myth}
\begin{proof}
    Пусть существуют $a, d, i$ такие, что $f^{(d)}_a[i] = 0$, тогда
    $C^{hd}_{hi}$ должно быть равно $-a C^{td}_{hi}$, а $C^{hd}_{ti}$ должно
    быть равно $-a C^{td}_{ti}$.
    \begin{equation*}
        \begin{cases}
            \binom{k-1}{i}(-d)^{k-1-i} & = \; 
                -a \sum\limits_{j=i}^{k-2} e \binom{j}{i} (-d)^{j-i} \\
            \sum\limits_{j=i}^{k-2} \binom{j}{i} (-d)^{j-i} & = \;
                -a \binom{k-1}{i}(-d)^{k-1-i} 
        \end{cases}
    \end{equation*} 
    $\sum\limits_{j=i}^{k-2} \binom{j}{i} (-d)^{j-i} \neq 0$ так как
    $\binom{k-1}{i}(-d)^{k-1-i} \neq 0$.
    Следовательно 
    \[
        a^{-1} \sum_{j=i}^{k-2} \binom{j}{i} (-d)^{j-i} =
            a \sum_{j=i}^{k-2} e \binom{j}{i} (-d)^{j-i}
    \]

    Значит $e = (a^{-1})^2$, что противоречит с тем, что $e$ -- квадратичный
    невычет по модулю $k$.
\end{proof}

\section{Гипотеза}
Также были рассмотрены следующие функции $h$ и $t$ одной переменной:
\[
    \begin{array}{l}
        h = ex^{k-1} + (x+1)^{k-1} \\
        t = x^{k-1} + e(x+1)^{k-1}
    \end{array}
\]

Возможно, доказать для этих функций будет проще, но пока мне не удалость это
сделать. Зато было получено, что эти функции задают сложные системы для всех
простых $k$ от 5 до 79, если $e \in [2 .. k-2]$. 
\end {document}
