\documentclass[12pt, a4paper]{article}
\usepackage[utf8x]{inputenc}
\usepackage[english,russian]{babel}
\usepackage[T2A]{fontenc}
\usepackage{cite}
\usepackage{url}
\usepackage{hyperref}
\usepackage{graphicx}
\usepackage{listings}
\usepackage[nottoc,notlot,notlof]{tocbibind}
\usepackage{icomma}
\usepackage {graphicx}
\let\stdsection\section
\renewcommand\section{\newpage\stdsection}
\addto\captionsrussian{% Replace "english" with the language you use
  \renewcommand{\contentsname}%
    {Оглавление}%
}
\hypersetup{
    colorlinks,
    allcolors=blue
}

\begin {document}

\thispagestyle {empty}

\begin {center}
\ \vspace{-4cm}

\includegraphics [width = 0.5 \textwidth] {msu.png} \\
{\scshape Московский Государственный Университет} \\
Факультет вычислительной математики и кибернетики\\

\vspace {5cm}

{\LARGE Реферат}

\vspace {1cm}

{\Huge \bfseries
<<Джон Нэш и его вклад в теорию игр>> \\}
\end {center}

\vfill
\vfill

\begin {flushright}
  \large
  Гордеев Михаил \\
  студент группы 618/1 \\

  \vspace {5mm}
\end {flushright}

\vfill

\begin {center}
Москва, 2016
\end {center}

\enlargethispage {4 \baselineskip}

\tableofcontents

\section{Биография}
Джон Нэш родился 13 июня 1928 года в Блюфилде, штат Западная Виргиния,
в строгой протестантской семье. Отец работал инженером-электриком в компании
Appalachian Electric Power, мать до замужества успела 10 лет проработать
школьной учительницей. В школе учился средне, а математику вообще не 
любил — в школе её преподавали скучно. Когда Нэшу было 14 лет, к нему в руки 
попала книга Эрика Т. Белла «Творцы математики». «Прочитав эту книгу, я сумел 
сам, без посторонней помощи, доказать малую теорему Ферма», — пишет Нэш в 
автобиографии.

\subsection{Учеба}
После школы последовала учёба в Политехническом институте Карнеги
(ныне частный Университет Карнеги-Меллона), где Нэш пробовал изучать химию,
прослушал курс международной экономики, а потом окончательно утвердился в
решении заняться математикой. В 1947 году, окончив институт с двумя
дипломами — бакалавра и магистра, — он поступил в Принстонский университет.
Институтский преподаватель Нэша Ричард Даффин снабдил его одним из самых
лаконичных рекомендательных писем. В нём была строчка: «Он — гений математики».

\subsection{Работа}
В Принстоне Джон Нэш услышал о теории игр, в ту пору только представленной
Джоном фон Нейманом и Оскаром Моргенштерном. Теория игр поразила его
воображение, да так, что в 20 лет Джон Нэш сумел создать основы научного
метода, сыгравшего огромную роль в развитии мировой экономики. В 1949 году
21-летний учёный написал диссертацию о теории игр. Сорок пять лет спустя он
получил за эту работу Нобелевскую премию по экономике «за фундаментальный
анализ равновесия в теории некооперативных игр».

В 1950—1953 годах Нэш опубликовал четыре революционные работы в области игр с
ненулевой суммой. Он обнаружил возможность «некооперативного равновесия», при
которой обе стороны используют стратегию, приводящую к устойчивому равновесию.
Этот результат получил впоследствии название «равновесие Нэша».

В 1951 году поступил на работу в Массачусетский технологический институт (МТИ).
Написал ряд статей по вещественной алгебраической геометрии и теории римановых
многообразий, которые были высоко оценены современниками.

В 1954 был арестован полицией Санта-Моники за непристойное поведение в мужской
раздевалке на пляже. Обвинение вскоре было снято, но Нэш был лишен допуска
к секретным проектам в корпорации RAND, где он работал консультантом по
совместительству.

\subsection{Болезнь}
Вскоре Джон Нэш встретил студентку, колумбийскую красавицу Алисию Лард, и в
1957 году они поженились. В июле 1958 года журнал Fortune назвал Нэша
восходящей звездой Америки в «новой математике». Вскоре жена Нэша забеременела,
но это совпало с болезнью Нэша — у него появились симптомы шизофрении. В это
время Джону было 30 лет, а Алисии — 26. Алисия пыталась скрыть всё происходящее
от друзей и коллег, желая спасти карьеру Нэша. Ухудшение состояния мужа все
сильнее угнетало Алисию. В 1959 году он лишился работы. Через некоторое время
Нэш был принудительно помещён в частную психиатрическую клинику в пригороде
Бостона, McLean Hospital, где ему поставили диагноз «параноидная шизофрения» и
подвергли психофармакологическому лечению. Адвокату Нэша удалось добиться его
освобождения из госпиталя через 50 дней. После выписки Нэш решил уехать в
Европу. Алисия оставила новорождённого сына у своей матери и последовала за
мужем. Нэш пытался получить статус политического беженца во Франции,
Швейцарии и ГДР и отказаться от американского гражданства. Биограф Сильвия
Назар сообщает, что в марте 1960 года Нэш посетил Лейпциг и проживал
несколько дней в семье Турмеров, пока власти принимали решение о его статус
. Наконец властям США удалось добиться возвращения Нэша — он был арестован
французской полицией и депортирован в США. По возвращении они обосновались
в Принстоне, где Алисия нашла работу. Но болезнь Нэша прогрессировала: он
постоянно чего-то боялся, говорил о себе в третьем лице, писал бессмысленные
почтовые карточки, звонил бывшим коллегам. Они терпеливо выслушивали его
бесконечные рассуждения о нумерологии и состоянии политических дел в мире.

В январе 1961 года полностью подавленная Алисия, мать Джона и его сестра Марта
поместили Джона в Trenton State Hospital в Нью-Джерси, где Джон прошёл курс
инсулиновой терапии. После выписки коллеги Нэша из Принстона решили ему помоч,
предложив ему работу в качестве исследователя, однако Джон опять
отправился в Европу, но на этот раз один. Домой он отправлял только загадочные
письма. В 1962 году, после трёх лет смятения, Алисия развелась с Джоном. При
поддержке матери она вырастила сына сама. Впоследствии у него также
развилась шизофрения.

Коллеги-математики продолжали помогать Нэшу — они дали ему работу в
университете и устроили встречу с психиатром, который выписал антипсихотические
лекарства. Состояние Нэша улучшилось, и он стал проводить время с Алисией и
своим первым сыном Джоном Дэвидом. «Это было очень обнадёживающее время, —
вспоминает сестра Джона Марта. — Это был достаточно долгий период. Но затем
все стало меняться». Джон перестал принимать лекарства, опасаясь, что они могут
вредить мыслительной активности, и симптомы шизофрении опять проявились.

В 1970 году Алисия Нэш, будучи уверенной, что, предав мужа, совершила ошибку,
приняла его вновь и это, возможно, спасло учёного от состояния бездомности. В
последующие годы Нэш продолжал ходить в Принстон, записывая на досках странные
формулы. Студенты Принстона прозвали его «Фантомом». Затем в 1980-х годах
Нэшу стало заметно лучше — симптомы отступили и он стал более вовлечённым в
окружающую жизнь. Болезнь, к удивлению врачей, стала отступать. На самом дел
, Нэш стал учиться не обращать на неё внимания и вновь занялся математикой.

\subsection{Признание}
11 октября 1994 года, в возрасте 66 лет, Джон Нэш получил Нобелевскую премию
в области экономики «За анализ равновесия в теории некооперативных игр».

Однако он был лишён возможности прочитать традиционную Нобелевскую лекцию в
Стокгольмском университете, так как организаторы опасались за его состояние.
Вместо этого был организован семинар (с участием лауреата), на котором
обсуждался его вклад в теорию игр. После этого Джон Нэш всё же был приглашён
прочитать лекцию в другом университете — Уппсалы. По словам приглашавшего
его профессора Математического института университета Уппсалы Кристера
Кисельмана, лекция была посвящена космологии.

В 2001 году, через 38 лет после развода, Джон и Алисия вновь поженились. Нэш
вернулся в свой офис в Принстоне, где продолжал заниматься математикой.

В 2008 году Джон Нэш выступил с докладом на тему «Ideal Money and
Asymptotically Ideal Money» на международной конференции Game Theory and
Management в Высшей школе менеджмента Санкт-Петербургского
государственного университета.

В 2015 году Джон Нэш получил высшую награду по математике — Абелевскую премию
за вклад в теорию нелинейных дифференциальных уравнений.

Примечательный факт: получив и Нобелевскую, и Абелевскую премии — Джон Форбс
Нэш стал первым человеком в мире, который был удостоен обеих престижных наград.

\subsection{Гибель}
Джон Нэш погиб 23 мая 2015 года (в возрасте 86 лет) вместе с супругой,
Алисией Нэш (ей было 83 года), в автомобильной катастрофе в штате Нью-Джерси.
Водитель такси, в котором ехали супруги, потерял управление при обгоне и
врезался в разделительный барьер. Обоих непристегнутых пассажиров выбросило
наружу при ударе, и приехавшие медики констатировали смерть на месте
происшествия.

\section{Теория игр}
Сознательно или несознательно люди играют в игры, причем в масштабах наций и
культур. Игра - это культурный феномен.
Представляется, что игра – это вечный культурный феномен, который сопровождает
человека всю жизнь. Люди всего мира играют в игры независимо от нации, цвета
кожи, вероисповедания, возраста и т.п.

Известный французский социолог Р. Кайуа писал: «Игра есть форма активности,
свободной, изолированной, нечеткой (т.е. те правила, по которым она
проводится, предоставляют определенную свободу действия), непродуктивной
(т.е. в процессе игры не создается никаких новых материальных благ),
регламентированной, т.е. протекающей по определенным правилам и фиктивно
, т.е. сопровождающейся особым сознанием иной реальности»\cite{ka07}.

Игра есть форма активности. В игры играют свободные люди по собственному
свободно выраженному желанию. В отличие от трудовой регламентированной
деятельности, к которой человек, так или иначе принужден цивилизацией, игра
есть проявление свободы.

Следует отметить, что в теорию игр внесли вклад многие исследователи. 
Игру как один из феноменов человеческого бытия многократно пытались
осмыслить представители самых различных научных школ.

Феномен игры представляет интерес для исследования, прежде всего, в таких
аспектах, как соотношение игры и искусства, роль игровой деятельности в
процессе воспитания и социализации личности, соотношение игры и культуры
вообще.

Интерес к исследованию игры как культурному феномену, поиски его сущности,
моделирование игры как системы стратегий, попытки вывести общее составляюще
, объединяющее столь различные проявления человеческой деятельности,
обозначенных в языке понятием «игра», способствовали формированию
самостоятельного исследовательского направления - «игрологии»\cite{ap07}.

Большой вклад в развитие теории игр, как ранее отмечалось, внес Нобелевский
лауреат Дж. Нэш. он окончил институт
Карнеги в 1948 году (ныне называется университет Карнеги-Меллон), получил
докторскую степень в области философии в 1950 году в Принстоне. В течение
пятилетнего периода, начиная с его докторского тезиса в 1949 году, он
установил математические принципы современной теории игр.

В четырех книгах, изданных между 1950-53 годами, Дж. Нэш описал несовместимую
теорию игр и теорию равновесия. В этот период Нэш опубликовал четыре без
преувеличения революционные работы, в которых представил глубокий анализ
«игр с ненулевой суммой» - особого класса игр, в которых все участники или
выигрывают, или терпят поражение. Примером такой игры могут стать
переговоры об увеличении зарплаты между профсоюзом и руководством компании.
Эта ситуация может завершиться либо длительной забастовкой, в которой
пострадают обе стороны, либо достижением взаимовыгодного соглашения. Нэш сумел
разглядеть новое лицо конкуренции, смоделировав ситуацию, впоследствии
получившую название «равновесие по Нэшу» или «некооперативное равновесие»,
при которой обе стороны используют идеальную стратегию, что и приводит к
созданию устойчивого равновесия. Игрокам выгодно сохранять это равновесие,
так как любое изменение только ухудшит их положение\cite{vm03}.

Ученый начал исследовать то, что сам назвал «умственными беспорядками» в
1959 году и оставался в уединении в течение последующих 30 лет, страдая от
параноидальной шизофрении.

Джон Нэш возвратился к своим академическим исследованиям, как только болезнь
была вылечена, и в 1994 году он получил Нобелевскую премию в области экономики
за значительную работу, проведенную при исследовании математической теории
игры. В 1997 году Джон Нэш написал «Эссе о Теории Игр».

Какой вклад внес Джон Нэш в теорию игры?
Первые концепции теории игр анализировали антагонистические игры, когда
есть проигравшие и выигравшие за их счет игроки.

Игры представляют собой строго определённые математические объекты. Игра
образуется игроками, набором стратегий для каждого игрока и указаниями
выигрышей, или платежей игроков для каждой комбинации стратегий. Большинство
кооперативных игр описываются характеристической функцией, в то время как
для остальных видов чаще используют нормальную или экстенсивную форму.
Характеризующие признаки игры как математической модели ситуации:
\begin{itemize}
\item наличие нескольких участников;

\item неопределенность поведения участников, связанная с наличием у каждого
    из них нескольких вариантов действий;

\item различие (несовпадение) интересов участников;

\item взаимосвязанность поведения участников, поскольку результат, получаемый
    каждым из них, зависит от поведения всех участников;

\item наличие правил поведения, известных всем участникам.
\end{itemize}

Как уже указывалось, Д.Нэш разрабатывает методы анализа, в которых все
участники или выигрывают, или терпят поражение. Эти ситуации получили названия
«равновесие по Нэшу», или «некооперативное равновесие», в ситуации стороны
используют оптимальную стратегию, что и приводит к созданию устойчивого
равновесия. Игрокам выгодно сохранять это равновесие, так как любое изменение
ухудшит их положение.

Эти работы Дж. Нэша стали серьёзным вкладом в развитие теории игр, так как
были пересмотрены математические инструменты экономического моделирования.

Дж. Нэш показывает, что классический подход к конкуренции А.Смита, когда
каждый сам за себя, неоптимален. Более оптимальны стратегии, когда каждый
старается сделать лучше для себя, делая лучше для других.

Хотя теория игр первоначально и рассматривала экономические модели, вплоть
до 1950-х она оставалась формальной теорией в рамках математики. Но уже с
1950-х гг. начинаются попытки применить методы теории игр не только в
экономике, но в биологии, кибернетике, технике, антропологии. Во время
Второй мировой войны и в послевоенный период теорией игр серьёзно
заинтересовались военные, которые увидели в ней мощный аппарат для
исследования стратегических решений.

В 1960 - 1970 гг. интерес к теории игр угасает, несмотря на значительные
математические результаты, полученные к тому времени.

С середины 1980-х гг. начинается активное практическое использование теории
игр, особенно в экономике и менеджменте.

Следует отметить, что за последние 20 - 30 лет значение теории игр и
интерес значительно растет, некоторые направления современной экономической
теории невозможно изложить без применения теории игр.

Тем временем теория игр стала важнейшим инструментом в бизнесе и
экономической теории. И все исследования в этой области опирались на
знаменитую работу Нэша.

Нэш получил Нобелевскую премию по экономике. Обычно нобелевские лауреаты
где-то в закоулках имеют некоторую политическую поддержку. Однако все,
что имел Нэш, было его личной заслугой. Нэш перенес то, что казалось
сродни смерти, по крайней мере, духовной, социальной. Нэш преодолел,
казалось бы, непреодолимое. Еще и оказался первым человеком получившим и
Нобелевскую и Абелевскую премии.

\section{Заключение}
Мир, окружающий человека, разнообразен и сохраняет в себе возможность
усложняться все больше и больше.

Коль скоро игра является преобразованием этого мира, то мы и будем
сталкиваться с все большим разнообразием игр или иных видов деятельности,
называемых в конечном итоге игрой.
В этом бесконечном движении или движении к бесконечности человек ищет
некоторые пространственно-временные ориентиры.

На мой взгляд, вклад Д. Нэша в теорию игр актуален и сегодня, когда каждое
количество игр возросло в десятки раз в сравнении с тем временем, когда он
создавал свои умозаключения.

Игру как один из феноменов человеческого бытия многократно пытались осмыслить
представители самых различных научных школ.
Феномен игры представлял интерес для исследования в разных аспектах.

Интерес к исследованию игры как культурному феномену, поиски его сущности,
моделирование игры как системы стратегий, попытки вывести общее составляюще
, объединяющее столь различные проявления человеческой деятельности,
обозначенных в языке понятием «игра», способствовали формированию
самостоятельного исследовательского направления - «игрологии».
Джон Нэш внес большой вклад в развитие теории игр. 

\makeatletter
\renewcommand*{\@biblabel}[1]{\hfill#1.}
\makeatother

\begin{thebibliography}{0}
    \bibitem{ka07} Кайуа Р. Игры и люди. Статьи и эссе по социологии культуры.
        М.: Издательство: ОГИ, 2007. - С.18 
    \bibitem{ap07} Апинян Т. Игра в пространстве серьезного: Игра, миф, ритуал,
        сон, искусство и другие. СПб.: Изд-во СПбГУ, 2007. – С.56 
    \bibitem{vm03} А. А. Васин, В. В. Морозов. Введение в теорию игр с 
        приложениями к экономике. М.: 2003.
    \bibitem{wiki} {\it ru.wikipedia.org/wiki/Нэш,\_Джон\_Форбс}
\end{thebibliography}

\end{document}
