\documentclass [12pt, a4paper] {article}
\usepackage[utf8x]{inputenc}
\usepackage[english,russian]{babel}
\usepackage[T2A]{fontenc}
\usepackage {graphicx}
\usepackage{ amssymb, latexsym, amsmath, amsthm}

\xdef\LastDeclaredEncoding{T2A}
\begin{document}

\thispagestyle {empty}

\begin {center}
\vspace{-4cm}

\includegraphics [width = 0.5 \textwidth] {msu.png} \\
{\scshape Московский Государственный Университет} \\
Факультет вычислительной математики и кибернетики\\

\vspace {5cm}
{\LARGE <<Введение в диплом>> }

\vspace {1cm}

{\Huge \bfseries
<<О сложности систем функций некоторых многозначных логик в классе
    поляризованных полиномов>> \\}
\end {center}

\vfill
\vfill

\begin {flushright}
  \large
  Гордеев Михаил \\
  студент группы 618/1 \\

  \vspace {5mm}
\end {flushright}

\vfill

\begin {center}
Москва, 2016
\end {center}

\enlargethispage {4 \baselineskip}

\newpage
\section{Введение}

Одним из стандартных способов задания функций k\nobreakdash-значной логики
являются поляризованные полиномиальные формы (ППФ), которые также называются
обобщенными формами Рида-Мюллера, или каноническими поляризованными полиномами.
В ППФ каждая переменная имеет определенную поляризацию.  Длиной полиномиальной
формы называется число попарно различных слагаемых в ней.  Длиной функции $f$ в
классе ППФ называется наименьшая длина среди длин всех поляризованных
полиномиальных форм, реализующих $f$. Функция Шеннона $L^K_k(n)$ длины
определяется как наибольшая длина среди всех функций $k$\nobreakdash значной
логики в классе $K$ от~$n$~переменных, если $K$ опущено, то подразумевается
класс ППФ. Практическое применение ППФ нашли при построении программируемых
логических матриц (ПЛМ)~\cite{ue04, sb90}, сложность ПЛМ напрямую зависит от
длины ППФ, по которой она построена. Поэтому в ряде работ исследуется сложность
ППФ различных функций \cite{sv93,pn95,ss02,kk05,sd08,mn12,sm09}. Также
рассматривают системы функций. Сложностью $L_k^\text{ППФ}(F)$ системы функций
$k$-значной логики $F = \{f_1(x_1, \ldots, x_n), \ldots, f_m(x_1, \ldots,
x_n)\}$ в классе ППФ называется минимальная сложность среди всех таких систем
ППФ $\{p_1, \dots, p_m\}$, что все ППФ $p_1, \dots, p_m$ имеют один и тот же
вектор поляризации , и ППФ $p_j$ реализует функцию $f_j$, $j=1 \dots m$.
Понятно, что для произвольной системы функций алгебры $k$-значной логики $F =
\{f_1(x_1, \ldots, x_n), \ldots, f_m(x_1, \ldots, x_n)\}$ верна оценка
$L_k^\text{ППФ}(F) \leqslant k^n$. В данной работе получена нижняя оценка
$L_k^\text{ППФ}(F) \geqslant k^n$ для некоторых $k > 3$.

В 1993  В.\,П.\,Супрун~\cite{sv93} получил первые оценки функции Шеннона для
функций алгебры логики :
$$
L_2(n) \geqslant C_n^{[\frac{n}{2}]},
$$
$$
L_2(n) < 3 \cdot 2^{n-1},
$$
где [$a$] обозначает целую часть $a$.

Точное значение функции Шеннона для функций алгебры логики в 1995\,г. было
найдено Н.\,А.\,Перязевым~\cite{pn95} :
$$
L_2(n) = \left[\frac{2^{n+1}}{3}\right].
$$

Функции $k$\nobreakdash-значных логик являются естественным обобщением функций
алгебры логики.  Для функций $k$\nobreakdash-значной логики верхняя оценка
функции Шеннона была получена в 2002\,г. С.\,Н.\,Селезневой~\cite{ss02} :
$$
L_k(n) < \frac{k(k-1)}{k(k-1)+1}k^n,
$$
в 2015 году она была улучшена \cite{by15}
$$
L_k(n) < \frac{k(k-1)-1}{k(k-1)}k^n.
$$


При построении ПЛМ рассматривают и другие полиномиальные формы. Например класс
обобщенных полиномиальных форм.  В классе обобщенных полиномиальных форм, в
отличие от класса поляризованных полиномиальных форм, переменные могут иметь
различную поляризацию в разных слагаемых. В статье
К.\,Д.\,Кириченко~\cite{kk05}, опубликованной в 2005\,г., был предложен метод
построения обобщенных полиномиальных форм из которогоследует
$L^{\text{О.П.}}_2(n) = O(\frac{2 ^ n}{n})$.

В работах \cite{sd08, bs14} получено, что
$L^{\text{О.П.}}_k(n) = O(\frac{k ^ n}{n})$.

В 2012\,г. Н.\,К.\,Маркеловым была получена нижняя оценка функции Шеннона для
функции трехзначной логики в классе поляризованных полиномов~\cite{mn12}:
$$
L_3(n) \geqslant \left[\frac{3}{4}3^n\right],
$$
в \cite{ss15} эта оценка была достигнута на симметрических функциях.
Также в работе \cite{ss15} были получены следующие оценки:
$$
L_2^\text{ППФ}(F) \geqslant 2^n \text{,}
$$
$$
L_3^\text{ППФ}(F) \geqslant 3^n \text{.}
$$


\makeatletter
\renewcommand*{\@biblabel}[1]{\hfill#1.}
\makeatother

\newpage

\begin{thebibliography}{0}
\bibitem{ue04} Угрюмов~Е.\,П. Цифровая схемотехника. СПб.: БХВ-Петербург, 2004.
\bibitem{sb90} Sasao T., Besslich P. On the complexity of mod-2 sum PLA’s  //
    IEEE Trans.on Comput. 39. N 2. 1990. P.~262--266.
\bibitem{sv93} Супрун~В.\,П. Сложность булевых функций в классе канонических
    поляризованных полиномов // Дискретная математика. 5.
    \textnumero 2. 1993. С. 111--115.
\bibitem{pn95} Перязев Н.\,А. Сложность булевых функций в классе полиномиальных
    поляризованных~форм // Алгебра и логика. 34.
    \textnumero 3. 1995. С. 323--326.
\bibitem{ss02} Селезнева С.\,H. О сложности представления функций многозначных
    логик поляризованными полиномами. Дискретная
    математика. 14. \textnumero 2. 2002. С.~48--53.
\bibitem{kk05} Кириченко~К.\,Д. Верхняя оценка сложности полиномиальных
    нормальных форм булевых функций
    // Дискретная математика. 17. \textnumero 3. 2005. С. 80--88.
\bibitem{sd08} Селезнева С.\,Н. Дайняк А.\,Б. О сложности обобщенных
    полиномов k\nobreakdash-значных функций // Вестник Московского
    университета. Серия 15. Вычислительная математика и кибернетика. 
        textnumero 3. 2008. С. 34--39.
\bibitem{mn12} Маркелов Н.\,К. Нижняя оценка сложности функций трехзначной
    логики в классе поляризованных полиномов // Вестник
    Московского университета. Серия 15. Вычислительная математика и
        кибернетика. \textnumero 3. 2012. С. 40--45.
\bibitem{sm09} Селезнева С.\,H. Маркелов Н.\,К. Быстрый алгоритм построения
    векторов коэффициэнтов поляризованных полиномов
    k-значных функций // Ученые записки Казанского университета. Серия
        Физико-математические науки. 2009. 151.
    \textnumero 2 С.~147-151.
\bibitem{bs14} Башов~М.\,А., Селезнева~С.\,Н. О длине функций $k$-значной
    логики в классе полиномиальных нормальных
    форм по модулю $k$ // Дискретная математика. -- 2014. -- Т. 26,
        вып. 3. -- С. 3-9.
\bibitem{ss15} Селезнева С.\,H. Сложность систем функций алгебры логики и
    систем функций трехзначной логики в классах
    поляризованных полиномиальных форм // Дискретная математика. -- 2015. --
        Т. 27, вып. 1. -- С. 111 -- 122. 
\bibitem{by15} Балюк~А.\,С., Янушковский~Г.\,В. Верхние оценки функций над
    конечными полями в некоторых классах
    кронекеровых форм // Известия Иркутского государственного университета.
        Серия: Математика. -- 2015. -- Т.14. -- С. 3-17.
\end{thebibliography}

\end{document}
