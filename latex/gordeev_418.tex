\documentclass[usePics]{lomonosov}
\begin{thesis}
\Title%
{О длине некоторых периодических функций пятизначной логики в классе
поляризованных полиномиальных форм}{{Гордеев\,М.\,М.}}%
% {Гордеев Михаил Михайлович}%
% {gordmisha@gmail.com}%
% {Кафедра математической кибернетики}%
% {Научный руководитель: к.ф.-м.н., доц. Селезнева Светлана Николаевна}
\Author{Гордеев~Михаил~Михайлович}{Студент}{Факультет ВМК МГУ имени М.\,В.\,Ломоносова}{Москва}{Россия}{gordmisha@gmail.com}


Одним из стандартных способов задания функций k\nobreakdash-значной логики
являются поляризованные полиномиальные формы (ППФ). В ППФ каждая переменная
имеет определенную поляризацию. %Длиной полиномиальной формы называется число
попарно различных слагаемых в ней. Практическое применение ППФ нашли при
построении программируемых логических матриц (ПЛМ)~[1], сложность ПЛМ напрямую
зависит от длины ППФ, по которой она построена. Поэтому в ряде работ исследуется
сложность ППФ различных функций~[2-4].

%В работе используются следующие определения:

Пусть $k \geqslant 2$ -- натуральное число, $E_k = \{0, 1, \dots, k - 1\}$,
отображение $f^{(n)} : E_k^n \rightarrow E_k$ называется функцией $k$-значной
логики. Обозначим через $P_k^n$ множество всех функций $k$-значной логики,
зависящих от $n$ переменных. Поляризованным мономом $K^{\delta}$ по вектору
поляризации $\delta = (d_1, \dots, d_n) \in E_k^n$, назовем выражение вида
$(x_{i_1} + d_{i_1} )^{m_1}\cdots(x_{i_r} + d_{i_r})^{m_r}$. Поляризованная
полиномиальная нормальная форма (ППФ) по вектору поляризации $\delta$ -- это
выражение вида $\sum\limits_{i=1}^lc_i \cdot K^{\delta}_i$, где $c_i \in E_k
\backslash \{0\}$, и $K^{\delta}_i \neq K^{\delta}_j$ при $i \neq j$. Число $l$
называется длиной ППФ.

Известно, что при каждом простом $k$ каждая функция $k$-значной логики
$f(x_1,\dots,x_n)$ задается однозначной ППФ $P^{\delta}(f)$ по каждому вектору
поляризации $\delta \in E_k^n$. Пусть $l(P)$ обозначает длину ППФ $P$, $l(f)$
обозначает наименьшую длину среди всех ППФ, представляющих функцию $k$-значной
логики $f$. Введем функцию Шеннона $L_k(n)$ длины функций $k$-значной логики в
классе ППФ: $ L_k(n) = \max\limits_{f\in P_k^n} \min\limits_{\delta \in E_k^n}
l(P^{\delta}(f)). $ Перязев Н.\,А. в 1995 г. получил точное значение функции
Шеннона для функций алгебры логики: $L_2(n) = \left[\frac{2^{n+1}}{3}\right]$.
Селезнева С.\,Н. в 2002 г. нашла верхнюю оценку функции Шеннона для функций
$k$-значной логики при простых $k$: $L_k(n) < \frac{k(k-1)}{k(k-1)+1}k^n$.
Маркелов Н.\,К. в 2012 г. получил нижнюю оценку функции Шеннона для функций
трехзначной логики: $L_3(n) \geqslant \left[\frac{3}{4}3^n\right]$.

Функция $k$\nobreakdash-значной логики $f(x_1 ,\dots , x_n)$ называется
симметрической, если $f(\pi(x_1), \dots, \pi(x_n)) = f(x_1, \dots, x_n)$ для
произвольной перестановки $\pi$ на множестве переменных $\{x_1,\dots,x_n\}$.
Симметрическая функция $f_{\tau}^{(n)}$ называется периодической c периодом
$\tau = (\tau_0 \tau_1 \dots \tau_{T-1}) \in E_k^T$, если $f(\alpha) = \tau_j$
при $|\alpha| = j\pmod T$ для всех $\alpha = (a_1,\dots,a_n)\in E_k^n$,
$|\alpha| = \sum_{i=1}^na_i$. Пусть $T \geqslant  1$, $\Pi = \{\tau_1, \dots,
\tau_s | \tau_i \in E_k^T\}$, $A_{\Pi} = \{f_{\tau}^{(n)}|\tau \in \Pi, n
\geqslant 1\}$. Класс $A_{\Pi}$ называется вырожденным, если для всех периодов
$\tau \in \Pi$ верно, что $l(f_{\tau}^{(n)}) = \bar{o}(k^n)$ при $n\rightarrow
\infty$.

Рассмотрим периодические симметрические функции $f_n = f_{(1,1,4,4)}^{(n)} \in
P_5^n$ и $g_n=f_{(1,4,4,1)}^{(n)} \in P_5^n$ с периодами $(1,1,4,4)$ и
$(1,4,4,1)$ соответственно, $n \geqslant 1$. Введем класс $\mathcal{A}$ функций
пятизначной логики вида $a \cdot f_n + b \cdot g_n$, $a,b \in E_5$, $a \neq 0
\text{ или }b \neq 0$. И его подкласс $\mathcal{F}$, состоящий из функций $c
\cdot f_n, c \cdot g_n, c\cdot(f_n+g_n), c \cdot (f_n+4g_n), c \in \{1,2,3,4\}$.

В работе получены следующие результаты.

\textbf{Теорема 1.} \bigskip
\textit{
%Если $n \geqslant 1$
Если $n \geqslant 1$, $\varphi_n = f_n + 2\,g_n \in P_5^n$ или $\varphi_n =
f_n + 3\,g_n \in P_5^n$, то для любого вектора поляризации $\delta =
(d_1,\ldots,d_n) \in E_5^n$ верно $$ l(P^{\delta}(\varphi_n)) = 5^{n-m} \cdot
4^m ,$$ где $m = \begin{cases} \text{количество } \text{ <<4>> в векторе }
\delta, \text{ если } \varphi_n = f_n + 2\,g_n; \\ \text{количество } \text{
<<2>> в векторе } \delta, \text{ если } \varphi_n = f_n + 3\,g_n. \end{cases}$
}

\textbf{Теорема 2.} \bigskip \textit{ Если $n\geqslant 1$,
$\varphi_n(x_1,\dots,x_n) \in \mathcal{F}$ и $n$ четно, то $$ l(\varphi_n)
\leqslant 5^n\left(2\cdot\left(\frac{4}{5}\right)^{\frac{n}{2}} - \left(
\frac{4}{5} \right)^n\right).$$ }

\textbf{Следствие.} \bigskip
\textit{
Класс функций $\mathcal{A}$ является вырожденным.
}

\References
\begin{enumerate}
\item
    Sasao T., Besslich P.
    \emph{On the complexity of mod-2 sum PLA’s.}
    IEEE Trans.on Comput. 39. \textnumero 2. 1990. P.~262--266.

\item
    Перязев Н.\,А.
    \emph{Сложность булевых функций в классе полиномиальных
    поляризованных~форм.}
    Алгебра и логика. 34. \textnumero 3. 1995. С. 323--326.

\item
    Селезнева С.\,H.
    \emph{О сложности представления функций многозначных логик поляризованными
    полиномами.}
    Дискретная математика. 14. \textnumero 2. 2002. С.~48--53.
\item
    Маркелов Н.\,К.
    \emph{Нижняя оценка сложности функций трехзначной логики в классе
    поляризованных полиномов.}  Вестник Московского университета. Серия 15.
    Вычислительная математика и кибернетика.
    \textnumero 3. 2012. С. 40--45.
\end{enumerate}
\end{thesis}
