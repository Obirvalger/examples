\documentclass[a4paper, 14pt]{extarticle}
\parindent=0cm
\usepackage{hyperref}
\usepackage{amsmath}
\usepackage{listings}

\begin{document}
\lstset{language=C++} 
\begin{flushright}
Gordeev Mikhail
\end{flushright}

\bigskip

1. Make the substitution $1 + \frac{1}{u} = e^t$.
$$u = \frac{1}{e^t - 1} $$
$$(2u + 1)\ln\left(1 + \frac{1}{u}\right) - 2 = t\left(\frac{2}{e^t - 1} + 1\right) - 2 = \frac{e^t(t-2) + t + 2}{e^t - 1} $$
$u > 0 \Rightarrow t > 0 \Rightarrow e^t - 1 > 0$.

Now consider $e^t(t-2) + t + 2$. Taylor series expansion of exponential function is
$$e^t = 1 + \frac{t}{1!} + \frac{t^2}{2!} + \frac{t^3}{3!} + \ldots$$
Use the expansion in the equation.
$$(1 + \frac{t}{1!} + \frac{t^2}{2!} + \frac{t^3}{3!} + \ldots)(t-2) + t + 2 = $$
$$t - 2 + t^2 - 2t + \frac{t^3}{2} - t^2 + t + 2 + \ldots + \frac{t^n}{(n-1)!} - \frac{t^n}{n!/2} + \ldots$$
$$(n-1)! < n! / 2 \Rightarrow \frac{t^n}{(n-1)!} - \frac{t^n}{n!/2} > 0$$
The expression $(2u + 1)\ln\left(1 + \frac{1}{u}\right) - 2$ is positive for every $u>0$.

\bigskip

4. $X$ -- symmetric positive definite matrix $\Rightarrow$ $X = PDP^{-1}$, where $D$ -- diagonal matrix with
positive elements.

$X^{-n} = PD^{-n}P^{-1}$,
$(X^{n} + X^{2n})^{-1} = P(D^{n} + D^{2n})^{-1}P^{-1}$.

Trace of all expression is equal to trace of the inner diagonal matrix $D'$.
For all $i \in [1..m]$:
$$d'_i = d_i^{-n} - (d_i^n + d_i^{2n})^{-1} =
\frac{1}{d_i^n} - \frac{1}{d_i^n + d_i^{2n}} =
\frac{d_i^{2n} + d_i^n - d_i^n}{d_i^{2n} + d_i^{3n}} = \frac{1}{1+d_i^n}$$
Trace $= \frac{1}{1+d_1^n} + \frac{1}{1+d_2^n} + \ldots + \frac{1}{1+d_m^n}$.

\begin{equation*}
\begin{cases}
d_i < 1 & \Rightarrow \underset{n \rightarrow \infty}{d_i' \rightarrow 1} \\
d_i > 1 & \Rightarrow \underset{n \rightarrow \infty}{d_i' \rightarrow 0} \\
d_i = 1 & \Rightarrow d_i' = \frac{1}{2}  \\
\end{cases}
\end{equation*}

The answer can be any number $k/2$, where $k$ is integer in range $[0 .. 2m]$.

\bigskip

6. Operators {\tt+} and {\tt<<} have to return non void result.
Operator {\tt+} must be constant and return sum of its operands.
Operator {\tt<<} shuld use template stream, be a \emph{friend}
function and return stream. If use ostream then need to write
{\tt std::ostream}.

\begin{lstlisting}
class Complex
{
public:
  Complex(double real, double imaginary = 0)
  : _real(real),_imaginary(imaginary) {}

  Complex operator+(Complex other) const
  {
    Complex temp = *this;
    temp._real += other._real;
    temp._imaginary += other._imaginary;

    return temp;
  }
  
  template <typename T>
  friend T& ::operator<<(T& os, Complex x)
  {
    os<<"("<<x._real<<","<<x._imaginary<<")";

    return os;
  }

  Complex operator++()
  {
    ++_real;
    return *this;
  }

  Complex operator++(int)
  {
    Complex temp = *this;
    ++_real;
    return temp;
  }
private:
  double _real, _imaginary;
};
\end{lstlisting}

7. 
\begin{lstlisting}
bool is_balanced(BSTree *tree) {
  std::stack<std::pair<int, BSTree*>> st;
  st.push(std::make_pair(0,tree));
  int dmax = 0, dmin = tree->count();

  while (!st.empty()) {
    auto p = st.top();
    st.pop();
    int d = p.first;
    BSTree *t = p.second;
    if (t->is_leaf()) {
      dmin = std::min(dmin, d);
      dmax = std::max(dmax, d);
    } else {
      st.push(std::make_pair(d+1, t->left()));
      st.push(std::make_pair(d+1, t->right()));
    }
  }

  return abs(dmin - dmax) <= 1;
}
\end{lstlisting}
\end{document}
