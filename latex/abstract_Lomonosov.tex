%
%   Образец / Шаблон оформления тезиса
%
%
%   Если в тезисе каких-то разделов (картинок, списка литературы) нет, то соотвествующие команды надо закомментировать.
%   Файл для компиляции --- этот (example.tex, переименовый в фамилию автора, например, ivanov.tex).
%
%   ========================================================================================
%


%
%   Если в вашем документе нет картинок и вы хотите компилировать документ при помощи latex->dvips->ps2pdf, то уберите опцию usePics, заменив следующую строчку на
\documentclass{lomonosov}
% \documentclass[usePics]{lomonosov}

\begin{thesis}  % Сам тезис должен быть полностью помещен внутри окружения thesis

% Один автор
%\Title{Тема доклада}{{Иванов\,И.\,И.}}
% Несколько авторов
\Title{Сложные функции в классе поляризованных полиномиальных форм}{{Гордеев\,М.\,М.}}

%
%   Команда авторства. Выберете ту, что отвечает вашему тезису, и, если надо, раскомментируйте ее; остальные --- удалите или закомментируйте.
%

% Один автор
\Author{Гордеев~Михаил~Михайлович}{Студент}{Факультет ВМК МГУ имени М.\,В.\,Ломоносова}{Москва}{Россия}{gordmisha@gmail.com}

% Несколько авторв из одной организации
% \Author{Ильин Александр Владимирович, Шевцова Ирина Геннадьевна}{Математик, ассистент}{Факультет ВМК МГУ имени М.\,В.\,Ломоносова}{Москва}{Россия}{smu@cs.msu.ru, lomonosov@cs.msu.ru}

% Несколько авторов из разных организаций
%\AuthorM{{Иванов~Иван~Иванович}{Петров~Петр~Петрович}}{%
%   {Аспирант, факультет ВМК МГУ имени М.\,В.\,Ломоносова, Москва, Россия}{Младший научный сотрудник, Ленинградский кораблестроительный институт, Ленинград, СССР}}{ivanov@cmc.msu.ru, petrov@cmc.msu.su}

В работе рассматриваются представления $k$\nobreakdash-значных функций поляризованными
полиномами. В [4] предложено применение поляризованных полиномов булевых функций
при проектировании интегральных схем. В [1] найдены сложные в этом классе булевы
функции. В [2] понятие поляризованных полиномов перенесено на многозначные
функции. В [3] определена задача о сложности систем $k$\nobreakdash-значных функций в классе
поляризованных полиномов и найдены сложные в этом классе системы функций при
$k=2$ и $k=3$. В работе получены сложные в классе поляризованных полиномов
системы функций при каждом простом $k$.

В докладе также представляется система символьных преобразований для работы с
поляризованными полиномами. Эта система является библиотекой на языке
программирования Perl. Она позволяет складывать, умножать полиномы, строить для
функции поляризованный полином, переходить от одной поляризации полинома к
другой, находить сложность поляризованных полиномов. Результаты могут
представляться либо в виде pdf\nobreakdash-файла (для анализа и печати), либо в виде
csv\nobreakdash-файла (для последующей обработки с помощью электронных таблиц).

%
%   Список литературы, если он есть
%
\begin{references}

\Source  Перязев Н.\,А.
  Сложность булевых функций в классе полиномиальных поляризованных~форм
  // Алгебра и логика. 34. \textnumero 3. 1995. С. 323--326.

\Source Селезнева С.\,H.
  О сложности представления функций многозначных логик поляризованными
  полиномами // Дискретная математика. 14. \textnumero 2. 2002.
  С.~48--53.

\Source  Селезнева С.\,H. Сложность систем функций алгебры логики и систем
  функций трехзначной логики в классах поляризованных полиномиальных форм
  // Дискретная математика. -- 2015. -- Т. 27, вып. 1. -- С. 111 -- 122.

\Source  Sasao T., Besslich P. On the complexity of mod-2 sum PLA’s
  // IEEE Trans.on Comput. 39. \textnumero 2. 1990. P.~262--266.

\end{references}

\end{thesis} % Сам тезис должен быть полностью помещен внутри окружения thesis
