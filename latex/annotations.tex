\documentclass[a4paper, 14pt]{extarticle}
\parindent=0cm
\usepackage{hyperref}

\begin{document}
The text is entitled "Monad Plus".

The article is devoted to Monad Plus -- monads with additional laws.
It is spoken in detail about definition and examples.
Monad Plus could be used in parallel parsing, another approach is to use it to define guard function.
The fact that List and Maybe are in Monad Plus class is stressed.

\medskip
\url{https://en.wikibooks.org/wiki/Haskell/MonadPlus}

\vskip 1cm

The text is entitled "All About Monads".

The paper is concerned with monads.
The text gives a detailed account of definition of monads, describing its construction and
applications of monads in real programs.
Much attention is given to show internal structure of monads and why we need this concept.
The article has a lot of examples of using List, Maybe, IO, State and many other monads.
In the end of the paper it gives a detailed analysis of combinations of monads.

\medskip
\url{https://wiki.haskell.org/All_About_Monads}

\vskip 1cm

The next annotations are about chapters of the book:

"Thinking functionally with Haskell", Richard Bird

\bigskip

The chapter 1 is entitled "What is functional programming?".

The text is concerned with explanation of concept of functional programming using examples 
from math functions. It is considered in detail program which shows most common words in
the text. It is shown how to install The Haskell Platform to use Haskell on your computer.

\vskip 1cm

The chapter 2 is entitled "Expressions, types and values".

The text gives a detailed account of ghci. It explains Glasgow Haskell Compiler or ghc with
ghci -- Haskell interpreter. As the title implies the chapter describes expressions and
operators using in them. It also shows operator sections and lambda functions -- tools that
allow us to write shorter and more clear programs. Much attention is given to Haskell type
structure with type inference system.

\vskip 1cm

The chapter 3 is entitled "Numbers".

As the title implies the chapter describes numbers where are Integral (Int, Integer) and
Floting (Float, Double) number types in Haskell. It gives a detailed analysis of Num 
type class, which helps us to discribe numeric typs i.e. types whith $+, -, *$ operations.
It is shown the whole hierarchy of Haskell numeric data types.

\vskip 1cm

The chapter 4 is entitled "Lists".

The chapter is concerned with lists. It provides list data type definition and shows three
types of lists: finite, partial, infinite. The main idea of the chapter is list constructors
which includes enumerators and list comprehensions. Then it draws our attention to main
list functions: map, filter, zipWith, which help us replace cycles from imperative
languages in Haskell.

\vskip 1cm

The chapter 5 is entitled "A simple Sudoku solver".

As the title implies the chapter describes how to write Sudoku solver whith Haskell.
The text gives a detailed analysis of data types using for describe cells and board.
It is shown how to write Sudoku solver, using composition of simple functions, 
working with lists.

\vskip 1cm

The chapter 6 is entitled "Proofs".

The main idea of the article is that list functions are the same as induction.
The article begins with describing induction over natural numbers. Than it shows
induction over lists: finite, partial and infinite.In conclusion the author proves
correctness of some function definitions, using folding functions.

\vskip 1cm

The chapter 7 is entitled "Efficiency".

The chapter is concerned with efficiency of program evaluation and computations.
It gives a detailed analysis of lazy evaluation -- evaluation strategy used in
Haskell by default. Much attention is given to controlling space and time used
by program. It shows in detail ghc's profiling tools, but it also noted that
the best way to improve a program’s performance is to use a better algorithm.

\vskip 1cm

The chapter 8 is entitled "Pretty-printing".

This chapter is devoted to an example of how to build a small library in Haskell.
The main idea of the text is to show steps of building pretty-printing library or
embedded domain specific language (EDSL). In conclusion the author says that the
growing success of Haskell is due to its ability to host a variety of EDSLs
without fuss.
\end{document}
